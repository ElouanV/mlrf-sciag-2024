\section{Dataset}

Dans le cadre de notre projet, nous avons utilisé le jeu de données CIFAR-10 pour entraîner et évaluer notre modèle. CIFAR-10 est un ensemble de données largement utilisé dans la communauté de l'apprentissage automatique pour la classification d'images.

Le jeu de données CIFAR-10 se compose d'un total de 60 000 images réparties en 10 classes distinctes. Chaque image a une taille de 32x32 pixels et est représentée en couleurs RGB, ce qui signifie qu'elle est composée de trois canaux : rouge, vert et bleu. Chaque canal contient des valeurs de pixel dans l'intervalle de 0 à 255.

Les 10 classes présentes dans CIFAR-10 sont les suivantes : avion, automobile, oiseau, chat, cerf, chien, grenouille, cheval, bateau et camion. Chaque classe compte 6 000 images, réparties également dans l'ensemble d'entraînement et l'ensemble de test.

CIFAR-10 est très populaire et couramment utilisé pour différentes raisons. Tout d'abord, il s'agit d'un ensemble de données bien établi et largement utilisé dans la communauté. De nombreux travaux de recherche et modèles de référence sont basés sur ce jeu de données, ce qui nous a permis de comparer nos résultats à ceux obtenus dans d'autres études.

De plus, CIFAR-10 offre une diversité d'images représentant différentes classes d'objets. Cela nous a permis de tester la capacité de notre modèle à reconnaître et classifier correctement différents types d'objets dans des images réelles.

Enfin, la taille des images dans CIFAR-10 (32x32 pixels) nous a permis de réduire la complexité de nos modèles et de diminuer les besoins en puissance de calcul. Cela nous a offert la possibilité d'itérer rapidement sur différentes paramètres d'apprentissage, ce qui était essentiel pour notre projet.